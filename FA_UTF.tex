\documentclass[12pt]{article}						%\documentclass[12pt]{article}
\usepackage{amsfonts,amsmath,amscd}				%\usepackage{amsfonts,amsmath,amscd}
\usepackage[utf8]{inputenc}						%\usepackage[cp1251]{inputenc}
\usepackage[english,russian]{babel}					%\usepackage[english,russian]{babel}
\usepackage{amssymb} 							%\usepackage{amssymb }
\usepackage{cite}
\usepackage{amssymb}
\usepackage{amsfonts}
\usepackage{amsmath}
\usepackage{amsthm}
\usepackage{graphicx,graphics,xcolor}
\usepackage{caption}%,psfrag}
\usepackage{subcaption}
\usepackage{xspace}
\textheight=252mm \textwidth=170mm \oddsidemargin=1mm
\topmargin=-55pt
\parindent=13mm
\pagestyle{myheadings}
\renewcommand{\baselinestretch}{1.33}				%\renewcommand{\baselinestretch}{1.33}
\newcommand{\im}{\mathop{\rm im}\nolimits}
\newcommand{\mes}{\mathop{\rm mes}\nolimits}
\renewcommand{\Re}{\mathop{\rm Re}\nolimits}
\renewcommand{\Im}{\mathop{\rm Im}\nolimits}
\newcommand{\cst}{\mathop{\rm const}\nolimits}
\newcommand{\dom}{\mathop{\rm dom}\nolimits}
\renewcommand{\span}{\mathop{\rm span}\nolimits}
\newtheorem{theorem}{Теорема}[subsection]
\newtheorem{lem}{Лемма}[subsection]
\newtheorem{zam}{Замечание}[subsection]%
\newtheorem{sled}{Следствие}[subsection]
\itemsep=10pt
\setcounter{equation}{0}
\renewcommand{\theequation}{\arabic{section}.\arabic{subsection}.\arabic{equation}}
%
\usepackage{graphicx}
\usepackage{fancyhdr} % пакет для установки колонтитулов
\pagestyle{fancy} % смена стиля оформления страниц
\fancyhf{} % очистка текущих значений
\fancyhead[C]{\thepage} % установка верхнего колонтитула
\renewcommand{\headrulewidth}{0pt} % убрать разделительную линию
\begin{document}

{\bf
\begin{center}
	Челябинский государственный университет)
\end{center}
\vspace*{4cm}

\vspace*{\baselineskip} \vspace*{\baselineskip}
\begin{center}
	ФУНКЦИОНАЛЬНЫЙ АНАЛИЗ
\end{center} {\large
\vspace*{\baselineskip}\vspace*{\baselineskip}\vspace*{\baselineskip}}
\thispagestyle{empty}
\begin{center}
	{\bf\it Практикум}
\end{center}
\vspace*{\baselineskip}

\vspace*{\baselineskip}

\vspace*{4cm}

\begin{center}
	Челябинск\\ 2021
\end{center}
}
%\end{titlepage}

\newpage

{\large
%\begin{titlepage}
%{\bf
\begin{center}
	Министерство  образования и науки\\ Российской
	Федерации\\\vspace*{\baselineskip}  Челябинский государственный университет
\end{center}

\vspace*{2cm}
}

\thispagestyle{empty} \vspace*{\baselineskip}
\vspace*{\baselineskip}
\begin{center}
	{\bf Функциональный анализ}
\end{center} {\large
\vspace*{\baselineskip}\vspace*{\baselineskip}\vspace*{\baselineskip}
\begin{center}
	{\it Учебно-методическое пособие}
\end{center}
\vspace*{\baselineskip}
\vspace*{\baselineskip}\vspace*{\baselineskip}\vspace*{\baselineskip}\vspace*{\baselineskip}\vspace*{\baselineskip}
\vspace*{\baselineskip}\vspace*{\baselineskip}\vspace*{\baselineskip}

\vspace*{\baselineskip}

\begin{center}
	Челябинск 2021
\end{center}

\newpage
\thispagestyle{empty} Одобрено С.М. Ворониным
\vspace*{\baselineskip}

	Методические указания содержат решения типовых задач по курсу основ функционального анализа, читаемого на третьем курсе математического факультета. В пособие включены краткая теория по каждому из разделов курса а также примеры контрольных заданий для проверки знаний, усвоенных на семинарских занятиях.

\vspace*{\baselineskip}
	Составители: \\
	доктор физ.-мат. наук, проф. В.Н.~Павленко\\
	доцент кафедры вычислительной матерматики В.А.~Адарченко\\
	ну и тут еще народ помогал ...\\
\vspace*{\baselineskip}

Рецензенты:

\newpage

\setcounter{page}{3}

\tableofcontents
\newpage

\section*{Предисловие}

	\textcolor{red}{Тут будут правильные слова о том, на кой фиг все это нужно, писать очередную методичку по функану}.
	\newpage

\section{Метрические пространства}

	\input{theory_01}
	\input{problems_01}
	\input{test_01}
	\newpage

\section{Нормированные пространства. Пространства со скалярным произведением}

	\input{theory_02}
	\input{problems_02}
	\input{test_02}
	\newpage

\section{Топология метрического пространства}

	\input{theory_03}
	\input{problems_03}
	\input{test_03}
	\newpage

\section{Предел последовательности в метрическом пространстве}

	\input{theory_04}
	\input{problems_04}
	\input{test_04}
	\newpage

\section{Предел функции и непрерывность. Сепарабельные метрические пространства}

	\input{theory_05}
	\input{problems_05}
	\input{test_05}
	\newpage

\section{Полные метрические пространства}

 	\input{theory_06}
 	\input{problems_06}
 	\input{test_06}
 	\newpage

\section{Теоремы о полных метрических пространствах.}

 	\input{theory_07}
 	\input{problems_07}
 	\input{test_07}
 	\newpage

\section{Компактные метрические пространства}

	\input{theory_08}
	\input{problems_08}
	\input{test_07}
	\newpage


\include{bible}

\newpage
{\large \thispagestyle{empty}
\begin{center}
{\bf Функциональный анализ} \end{center}
 \vspace*{\baselineskip}
 \begin{center}
{Практикум} \end{center}
 \vspace*{\baselineskip}
\begin{center}
Составители:\\
Воронин Сергей Михайлович,\\
Адарченко Владимир Анатольевич.
 \end{center}
}
 \vspace*{10mm}
 {\large
\vspace*{\baselineskip}\vspace*{\baselineskip}\vspace*{\baselineskip}
\begin{center}
Редактор
\end{center}
\vspace*{\baselineskip}

\begin{center}  Подписано в печать ...2021.\\ Формат 60x84
1 / 16.  Бумага типографская \textnumero 2.\\ Печать офсетная.
  Усл. печ. л. 2,0. Уч.-изд. л. 1,6.\\ Тираж
200 экз.
  Заказ 55 \end{center}

   \vspace*{\baselineskip}

\begin{center} Челябинский государственный университет\\
454021 Челябинск, ул. Братьев Кашириных, 129.\end{center}

 \vspace*{\baselineskip}

\begin{center} Полиграфический участок
 Издательского
центра ЧелГУ. \\
 454021 Челябинск, ул. Молодогвардейцев,
57б.\end{center}



\end{document}
